% Options for packages loaded elsewhere
\PassOptionsToPackage{unicode}{hyperref}
\PassOptionsToPackage{hyphens}{url}
\PassOptionsToPackage{dvipsnames,svgnames,x11names}{xcolor}
%
\documentclass[
  letterpaper,
  DIV=11,
  numbers=noendperiod]{scrartcl}

\usepackage{amsmath,amssymb}
\usepackage{lmodern}
\usepackage{iftex}
\ifPDFTeX
  \usepackage[T1]{fontenc}
  \usepackage[utf8]{inputenc}
  \usepackage{textcomp} % provide euro and other symbols
\else % if luatex or xetex
  \usepackage{unicode-math}
  \defaultfontfeatures{Scale=MatchLowercase}
  \defaultfontfeatures[\rmfamily]{Ligatures=TeX,Scale=1}
  \setmainfont[]{IBM Plex Sans}
\fi
% Use upquote if available, for straight quotes in verbatim environments
\IfFileExists{upquote.sty}{\usepackage{upquote}}{}
\IfFileExists{microtype.sty}{% use microtype if available
  \usepackage[]{microtype}
  \UseMicrotypeSet[protrusion]{basicmath} % disable protrusion for tt fonts
}{}
\makeatletter
\@ifundefined{KOMAClassName}{% if non-KOMA class
  \IfFileExists{parskip.sty}{%
    \usepackage{parskip}
  }{% else
    \setlength{\parindent}{0pt}
    \setlength{\parskip}{6pt plus 2pt minus 1pt}}
}{% if KOMA class
  \KOMAoptions{parskip=half}}
\makeatother
\usepackage{xcolor}
\setlength{\emergencystretch}{3em} % prevent overfull lines
\setcounter{secnumdepth}{-\maxdimen} % remove section numbering
% Make \paragraph and \subparagraph free-standing
\ifx\paragraph\undefined\else
  \let\oldparagraph\paragraph
  \renewcommand{\paragraph}[1]{\oldparagraph{#1}\mbox{}}
\fi
\ifx\subparagraph\undefined\else
  \let\oldsubparagraph\subparagraph
  \renewcommand{\subparagraph}[1]{\oldsubparagraph{#1}\mbox{}}
\fi


\providecommand{\tightlist}{%
  \setlength{\itemsep}{0pt}\setlength{\parskip}{0pt}}\usepackage{longtable,booktabs,array}
\usepackage{calc} % for calculating minipage widths
% Correct order of tables after \paragraph or \subparagraph
\usepackage{etoolbox}
\makeatletter
\patchcmd\longtable{\par}{\if@noskipsec\mbox{}\fi\par}{}{}
\makeatother
% Allow footnotes in longtable head/foot
\IfFileExists{footnotehyper.sty}{\usepackage{footnotehyper}}{\usepackage{footnote}}
\makesavenoteenv{longtable}
\usepackage{graphicx}
\makeatletter
\def\maxwidth{\ifdim\Gin@nat@width>\linewidth\linewidth\else\Gin@nat@width\fi}
\def\maxheight{\ifdim\Gin@nat@height>\textheight\textheight\else\Gin@nat@height\fi}
\makeatother
% Scale images if necessary, so that they will not overflow the page
% margins by default, and it is still possible to overwrite the defaults
% using explicit options in \includegraphics[width, height, ...]{}
\setkeys{Gin}{width=\maxwidth,height=\maxheight,keepaspectratio}
% Set default figure placement to htbp
\makeatletter
\def\fps@figure{htbp}
\makeatother

\KOMAoption{captions}{tableheading}
\makeatletter
\makeatother
\makeatletter
\makeatother
\makeatletter
\@ifpackageloaded{caption}{}{\usepackage{caption}}
\AtBeginDocument{%
\ifdefined\contentsname
  \renewcommand*\contentsname{Table of contents}
\else
  \newcommand\contentsname{Table of contents}
\fi
\ifdefined\listfigurename
  \renewcommand*\listfigurename{List of Figures}
\else
  \newcommand\listfigurename{List of Figures}
\fi
\ifdefined\listtablename
  \renewcommand*\listtablename{List of Tables}
\else
  \newcommand\listtablename{List of Tables}
\fi
\ifdefined\figurename
  \renewcommand*\figurename{Figure}
\else
  \newcommand\figurename{Figure}
\fi
\ifdefined\tablename
  \renewcommand*\tablename{Table}
\else
  \newcommand\tablename{Table}
\fi
}
\@ifpackageloaded{float}{}{\usepackage{float}}
\floatstyle{ruled}
\@ifundefined{c@chapter}{\newfloat{codelisting}{h}{lop}}{\newfloat{codelisting}{h}{lop}[chapter]}
\floatname{codelisting}{Listing}
\newcommand*\listoflistings{\listof{codelisting}{List of Listings}}
\makeatother
\makeatletter
\@ifpackageloaded{caption}{}{\usepackage{caption}}
\@ifpackageloaded{subcaption}{}{\usepackage{subcaption}}
\makeatother
\makeatletter
\@ifpackageloaded{tcolorbox}{}{\usepackage[many]{tcolorbox}}
\makeatother
\makeatletter
\@ifundefined{shadecolor}{\definecolor{shadecolor}{rgb}{.97, .97, .97}}
\makeatother
\makeatletter
\makeatother
\makeatletter
\@ifpackageloaded{fontawesome5}{}{\usepackage{fontawesome5}}
\makeatother
\ifLuaTeX
  \usepackage{selnolig}  % disable illegal ligatures
\fi
\IfFileExists{bookmark.sty}{\usepackage{bookmark}}{\usepackage{hyperref}}
\IfFileExists{xurl.sty}{\usepackage{xurl}}{} % add URL line breaks if available
\urlstyle{same} % disable monospaced font for URLs
\hypersetup{
  pdftitle={BEE 4750 Syllabus, Fall 2023},
  colorlinks=true,
  linkcolor={blue},
  filecolor={Maroon},
  citecolor={Blue},
  urlcolor={Blue},
  pdfcreator={LaTeX via pandoc}}

\title{BEE 4750 Syllabus, Fall 2023}
\usepackage{etoolbox}
\makeatletter
\providecommand{\subtitle}[1]{% add subtitle to \maketitle
  \apptocmd{\@title}{\par {\large #1 \par}}{}{}
}
\makeatother
\subtitle{Cornell University}
\author{}
\date{}

\begin{document}
\maketitle
\ifdefined\Shaded\renewenvironment{Shaded}{\begin{tcolorbox}[interior hidden, frame hidden, borderline west={3pt}{0pt}{shadecolor}, sharp corners, enhanced, boxrule=0pt, breakable]}{\end{tcolorbox}}\fi

\renewcommand*\contentsname{Table of contents}
{
\hypersetup{linkcolor=}
\setcounter{tocdepth}{2}
\tableofcontents
}
\hypertarget{course-information}{%
\subsection{Course Information}\label{course-information}}

\hypertarget{instructor}{%
\subsubsection{Instructor}\label{instructor}}

\begin{itemize}
\tightlist
\item
  \faIcon{user} {\href{https://viveks.me}{Vivek Srikrishnan}}
\item
  \faIcon{envelope}
  {\href{mailto:viveks@cornell.edu}{\nolinkurl{viveks@cornell.edu}}}
\item
  \faIcon{building} {318 Riley-Robb}
\item
  \faIcon{chalkboard-teacher} TBD
\end{itemize}

\hypertarget{ta}{%
\subsubsection{TA}\label{ta}}

\begin{itemize}
\tightlist
\item
  \faIcon{user} {TBD}
\item
  \faIcon{envelope} {TBD}
\item
  \faIcon{building} {TBD}
\item
  \faIcon{chalkboard-teacher} TBD
\end{itemize}

\hypertarget{meetings}{%
\subsubsection{Meetings}\label{meetings}}

\begin{itemize}
\tightlist
\item
  \faIcon{calendar} Mondays, Wednesdays
\item
  \faIcon{clock} 11:25-12:40
\item
  \faIcon{university} {105 Riley-Robb}
\end{itemize}

This is a 3-credit course which is required for the Environmental
Engineering major.

\hypertarget{course-description}{%
\subsubsection{Course Description}\label{course-description}}

Environmental processes have complicated dynamics and conflicting
objectives. These dynamics can complicate analyses and which focus on a
single component of the system, such as an individual pollution source.
In this course, we will adopt a systems approach to environmental
quality modeling and management, including applications in air and water
pollution control and solid waste management. In particular, we will:

\begin{itemize}
\tightlist
\item
  learn how to define systems and their boundaries;
\item
  simulate system dynamics using computer models;
\item
  formulate and solve linear and nonlinear optimization problems;
\item
  analyze and assess risk after introducing uncertainty;
\item
  make decisions under uncertainty with stochastic and dynamic
  programming; and
\item
  explore trade-offs across competing objectives.
\end{itemize}

\hypertarget{learning-objectives}{%
\subsection{Learning Objectives}\label{learning-objectives}}

At the end of this class, students will:

\begin{enumerate}
\def\labelenumi{\arabic{enumi}.}
\tightlist
\item
  Establish system boundaries and distinguish between exogenous and
  endogenous processes;
\item
  Develop mathematical models of environmental systems;
\item
  Determine strategies for managing systems using optimization;
\item
  Identify the trade-offs that result from competing objectives in
  environmental decision -making;
\item
  Analyze environmental system risk and vulnerabilities.
\end{enumerate}

\hypertarget{topics}{%
\subsection{Topics}\label{topics}}

\begin{itemize}
\tightlist
\item
  Introduction to environmental systems,
\item
  Modeling system dynamics
\item
  Multiple objectives and trade-offs
\item
  Uncertainty and risk (Monte Carlo analysis)
\item
  Dissolved oxygen in streams and rivers; waste load allocation (system
  simulation)
\item
  Modeling of watersheds \& lakes (defining objectives, constraints)
\item
  Modeling for air pollution control (model linearization; linear
  programming)
\item
  Location of waste disposal facilities (integer linear programming)
\item
  Robustness of solutions and sensitivity analysis
\end{itemize}

\hypertarget{prerequisites-preparation}{%
\subsection{Prerequisites \&
Preparation}\label{prerequisites-preparation}}

The following courses/material would be ideal preparation:

\begin{itemize}
\tightlist
\item
  Environmental Processes (BEE 2510 or BEE 2600)
\item
  Engineering Computation (ENGRD/CEE 3200)
\item
  One course in probability or statistics (ENGRD 2700, CEE 3040, or
  equivalent)
\end{itemize}

In the absence of one or more these prerequisites, you can seek the
permission of instructor.

If your programming or statistics skills are a little rusty, don't
worry! We will review concepts and build skills as needed.

\hypertarget{course-philosophy-and-expectations}{%
\subsection{Course Philosophy and
Expectations}\label{course-philosophy-and-expectations}}

The goal of our course is to help you gain competancy and knowledge in
the area of environmental systems analysis. This involves a dual
responsibility on the part of the instructor and the student. As the
instructor, my responsibility is to provide you with a structure and
opportunity to learn. To this end, I will commit to:

\begin{itemize}
\tightlist
\item
  provide organized and focused lectures, in-class activities, and
  assignments;
\item
  encourage students to regularly evaluate and provide feedback on the
  course;
\item
  manage the classroom atmosphere to promote learning;
\item
  schedule sufficient out-of-class contact opportunities, such as office
  hours;
\item
  allow adequate time for assignment completion;
\item
  make lecture materials, class policies, activities, and assignments
  accessible to students.
\end{itemize}

Students are encouraged to discuss any concerns with me during office
hours or through a course communications channel.

Students can optimize their performance in the course by:

\begin{itemize}
\tightlist
\item
  attending all lectures;
\item
  doing any required preparatory work before class;
\item
  actively participating in online and in-class discussions;
\item
  beginning assignments and other work early;
\item
  and attending office hours as needed.
\end{itemize}

\hypertarget{community}{%
\subsection{Community}\label{community}}

\hypertarget{diversity-and-inclusion}{%
\subsubsection{Diversity and Inclusion}\label{diversity-and-inclusion}}

Our goal in this class is to foster an inclusive learning environment
and make everyone feel comfortable in the classroom, regardless of
social identity, background, and specific learning needs. As engineers,
our work touches on many critical aspects of society, and questions of
inclusion and social justice cannot be separated from considerations of
systems analysis, objective selection, risk analysis, and trade-offs.

In all communications and interactions with each other, members of this
class community (students and instructors) are expected to be respectful
and inclusive. In this spirit, we ask all participants to:

\begin{itemize}
\tightlist
\item
  share their experiences, values, and beliefs;
\item
  be open to and respectful of the views of others; and
\item
  value each other's opinions and communicate in a respectful manner.
\end{itemize}

We all make mistakes in our communications with one another, both when
speaking and listening. Be mindful of how spoken or written language
might be misunderstood, and be aware that, for a variety of reasons, how
others perceive your words and actions may not be exactly how you
intended them. At the same time, it is also essential that we be
respectful and interpret each other's comments and actions in good
faith.

Please let me know if you feel any aspect(s) of class could be made more
inclusive. Please also share any preferred name(s) and/or your pronouns
with me if you wish: I use he/him/his, and you can refer to me either as
Vivek or Prof.~Srikrishnan.

\hypertarget{student-accomodations}{%
\subsubsection{Student Accomodations}\label{student-accomodations}}

Let me know if you have any access barriers in this course, whether they
relate to course materials, assignments, or communications. If any
special accomodations would help you navigate any barriers and improve
your chances of success, please exercise your right to those
accomodations and reach out to me as early as possible with your
\href{https://sds.cornell.edu/}{Student Disability Services} (SDS)
accomodation letter. This will ensure that we have enough time to make
appropriate arrangements.

If you need more immediate accomodations, but do not yet have a letter,
please let me know and then follow up with SDS.

\hypertarget{course-communications}{%
\subsubsection{Course Communications}\label{course-communications}}

Most course communications will occur via \href{https://edstem.org}{Ed
Discussion}. Public Ed posts are generally preferred to private posts or
emails, as other students can benefit from the discussions. If you would
like to discuss something privately, please do reach out through email
or a private Ed post (which will only be viewable by you and the course
staff).

Announcements will be made on the course website and in Ed. Emergency
announcements will also be made on Canvas.

\hypertarget{mental-health-resources}{%
\subsubsection{Mental Health Resources}\label{mental-health-resources}}

We all have to take care of our mental health, just as we would our
physical health. As a student, you may experience a range of issues
which can negatively impact your mental health. Please do not ignore any
of these stressors, or feel like you have to navigate these challenges
alone! You are part of a community of students, faculty, and staff, who
have a responsibility to look for one another's well-being. If you are
struggling with managing your mental health, or if you believe a
classmate may be struggling, please reach out to the course instructor,
the TA, or, for broader support, please take advantage of
\href{https://mentalhealth.cornell.edu/}{Cornell's mental health
resources}.

The TA and myself are not trained counselors, but we are here to support
you in whatever capacity we can. You should never feel that you need to
push yourself past your limits to complete any assignment for this class
or any other. If we need to make modifications to the course or
assignment schedule, you can certainly reach out to us, and all relevant
discussions will be kept strictly confidential.

\hypertarget{course-policies}{%
\subsection{Course Policies}\label{course-policies}}

\hypertarget{attendance}{%
\subsubsection{Attendance}\label{attendance}}

Attendance is not \emph{required}, but in general, students who attend
class regularly will do better and get more out of the class than
students who do not. Your class participation grade will reflect both
the quantity and quality of your participation, only some of which can
occur asynchronously. I will put as many course materials, such as
lecture notes and announcements, as possible online, but viewing
materials online is not the same as active participation and engagement.

Life happens, of course, and this may lead you to miss class. Let me
know if you need any appropriate arrangements ahead of time. For
example, please stay home if you're feeling sick! This is beneficial for
both for your own recovery and the health and safety of your classmates.
We will also make any necessary arrangements if whatever is going on
will negatively impact your grade, for example by causing you to be
unable to submit an assignment on time.

\hypertarget{mask-policies}{%
\subsubsection{Mask Policies}\label{mask-policies}}

\textbf{Masks are encouraged but not required in the classroom}, per
\href{https://covid.cornell.edu/updates/20220727-students-fall-semester.cfm}{university
policy}. However, the University \emph{strongly encourages} compliance
with requests to mask from students, faculty, and staff who are
concerned about the risk of infection. Please be respectful of these
concerns and requests if you cannot wear a mask.

While I understand that students often prefer their instructors to be
unmasked, I am frequently in contact with people whose underlying health
conditions make them higher-risk for severe complications from COVID-19.
As a result, I will not personally unmask unless every student in the
classroom is wearing their mask properly, to ensure that I can interact
with the entire class. Masks will be made available in class if you
would like to wear one and do not have one on you.

I will \textbf{require masks to be worn} in my office or during
in-person office hours, as we are necessarily interacting in close
quarters without great airflow.

\hypertarget{academic-integrity}{%
\subsubsection{Academic Integrity}\label{academic-integrity}}

\textbf{TL;DR}: Don't cheat, copy, or plagiarize!

This class is designed to encourage collaboration, and students are
encouraged to discuss their work with other students. However, I expect
students to abide by the
\href{http://theuniversityfaculty.cornell.edu/academic-integrity/}{Cornell
University Code of Academic Integrity} in all aspects of this class.
\textbf{All work submitted must represent the students' own work and
understanding}, whether individually or as a group (depending on the
particulars of the assignment). This includes analyses, code, software
runs, and reports. Engineering as a profession relies upon the honesty
and integrity of its practitioners (see \emph{e.g.} the
\href{https://www.asce.org/-/media/asce-images-and-files/career-and-growth/ethics/documents/asce-code-ethics.pdf}{American
Society for Civil Engineers' Code of Ethics}).

\hypertarget{external-resources}{%
\subsubsection{External Resources}\label{external-resources}}

The collaborative environment in this class should not be viewed as an
invitation for plagiarism. Plagiarism occurs when a writer intentionally
misrepresents another's words or ideas (including code!) as their own
without acknowledging the source. \textbf{All} external resources which
are consulted while working on an assignment should be referenced,
including other students and faculty with whom the assignment is
discussed. You will never be penalized for consulting an external source
for help and referencing it, but plagiarism will result in a zero for
that assignment as well as the potential for your case to be passed on
for additional disciplinary action.

\hypertarget{aiml-resource-policy}{%
\subsubsection{AI/ML Resource Policy}\label{aiml-resource-policy}}

As noted, all work submitted for a grade in this course must reflect
your own understanding. The use and consulation of AI/ML tools, such as
ChatGPT or similar, must be pre-approved and clearly referenced. If
approved, you must:

\begin{itemize}
\tightlist
\item
  reference the URL of the service you are using including the specific
  date you accessed it;
\item
  provide the exact query used to interact with the tool; and
\item
  report the exact response received.
\end{itemize}

Failure to attain prior approval or fully reference the interaction, as
described above, will be treated as plagiarism and referred to the
University accordingly.

\hypertarget{late-work-policy}{%
\subsubsection{Late Work Policy}\label{late-work-policy}}

In general, late work will be subjected to a 10\% penalty per day, which
can accumulate to 100\% of the total grade. However, sometimes things
come up in life. Your lowest homework will be dropped automatically.

While we will not automatically drop lab or application exercises,
please reach out \emph{ahead of time} if you have extenuating
circumstances (including University-approved absences or illnesses)
which would make it difficult for you to submit your work on time. Work
which would be late for appropriate reasons will be given extensions and
the late penalty will be waived.

\hypertarget{assessments}{%
\subsection{Assessments}\label{assessments}}

\hypertarget{application-exercises-5}\label{application-exercises-5}}

Some weeks, small exercises will be assigned to introduce or extend
concepts, or to get practice with programming syntax. Your notebooks for
these exercises should be submitted by the end of the given week, no
later than 9:00pm on Friday. These will be graded solely on the basis of
completion.

\hypertarget{lab-notebooks-15}\label{lab-notebooks-15}}

Some classes will involve hands-on exercises (which we will call
``labs'') which will give you guided practice applying the concepts and
methods from class. These classes will be announced on
\href{https://viveks.me/environmental-systems-analysis}{the course
website} ahead of time so anyone who is able can bring a laptop to
class, and notebooks will be provided on GitHub. These labs can be done
in groups; if you cannot bring a laptop to class for whatever reason,
you will be able to (and are encouraged to) work with other students,
though you must turn in your own notebook for grading.

Some details on lab logistics:

\begin{itemize}
\tightlist
\item
  Some of the labs may some time outside of class, but they will not be
  as intensive as a homework assignment.
\item
  You will be required to submit a PDF of your completed notebook to
  Gradescope by 9:00pm one week after the lab session. Tag the answers
  to each question: points will be deducted if this is not done.
\item
  While your lowest lab grade will not be dropped, late penalties will
  be waived for appropriate reasons discussed with the instructor
  (ideally ahead of time when circumstances allow).
\item
  Rubrics will be provided for lab grading as part of the lab
  assignments.
\end{itemize}

\hypertarget{homework-assignments-40}\label{homework-assignments-40}}

Approximately 5-6 homework assignments will be assigned throughout the
semester; the specifics depend on how quickly we move through the
material. You will typically have 10 days to 2 weeks to work on Students
are encouraged to collaborate and learn from each other on homework
assignments, but each student must submit their own solutions reflecting
their understanding of the material. Consulting and referencing external
resources and your peers is encouraged (engineering is a collaborative
discipline!), but plagiarism is a violation of academic integrity.

Some notes on assignment and grading logistics:

\begin{itemize}
\tightlist
\item
  Homework assignments will be distributed using GitHub Classroom.
  Students should make sure they update their GitHub repositories as
  they work on the assignments; this helps with answering questions and
  gives you a backstop in case something goes wrong and you can't submit
  your assignment on time.
\item
  Homeworks are due by 9:00pm Eastern Time on the designed due date.
  Your assignment notebook (which include your writeup and codes) should
  be submitted to Gradescope as a PDF with the answers to each question
  tagged (a failure to do this will result in deductions).
\item
  Rubrics will be provided for the homeworks as part of the
  assignments.s
\item
  Your lowest homework grade will be dropped. We can discuss
  arrangements if multiple assignments will be missed for
  university-approved reasons, preferably ahead of time.
\item
  Regrade requests for specific problems must be made within a week of
  the grading of that assignment. However, note that regrades can cut
  both ways: the TA can take away points as well!
\end{itemize}

\hypertarget{final-term-project-40}\label{final-term-project-40}}

This course will culminate with a term project with a topic selected
from a suggested list (provided mid-semester). The goal of this project
is to apply and extend the tools and approaches we will learn in class.
While we encourage drawing on other classes or interests when developing
and working on your project, \textbf{submitting work from another course
or work which was completed prior to the course is not permitted}.

The term project will be completed in small groups (2-3 students) for
students enrolled in BEE 4750 and individually for those in BEE 5750.
The final deliverable for this project will be a poster summarizing the
project and results. Ahead of that, you will submit the following:

\begin{itemize}
\tightlist
\item
  a proposal for feedback on the scope of your project; and
\item
  a 2-3 page report on the status and history of a regulation of
  interest relevant to the system you are studying.
\end{itemize}

Rubrics will be provided for the components of the project.

\hypertarget{covid-19-arrangements}{%
\subsection{COVID-19 Arrangements}\label{covid-19-arrangements}}

The particulars of how COVID-19 will affect class are fluid, depending
on Cornell policies and the state of any acute outbreaks. Let me know if
you will miss class due to quarantine (either official or self-imposed),
and we will make arrangements for streaming and recording class, as well
as any required virtual office hours or missed assignments. If class is
shifted online for any reason, we will make appropriate arrangements and
keep students informed. If we are not allowed to have in-person
meetings, all office hours will be moved online and we will figure out
alternatives to the group coding sessions.

In general, please err on the side of not attending class if you suspect
that you might be ill, with COVID-19 or anything else.



\end{document}
